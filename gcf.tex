\documentclass[12pt]{article}
\usepackage{graphicx}     %standard package. Note graphics<graphicx<epsfig
\usepackage{color}     %standard package
\usepackage{hyperref} %standard package written by Sebastian Rahtz
%\usepackage{subeqn}    %allows equations 1a, 1b
%\usepackage{subfig}    %allows figures 1a, 1b
\usepackage{amsmath}
\usepackage{amssymb}
\usepackage{url} 
\usepackage{bbm}%for indicator function
%\bibliographystyle{alpha}

\usepackage[nottoc]{tocbibind}

\usepackage[color,matrix,frame,arrow]{xy}
\usepackage[table,xcdraw]{xcolor}
\usepackage[shortlabels]{enumitem}
\usepackage{array}
\usepackage{pdflscape}
\usepackage{multicol, multirow}
\usepackage{ctable}
\usepackage{dirtree}
\usepackage{subfig}

%\usepackage{fdsymbol}
\usepackage{animate}% play animation

\usepackage[vlined,ruled]{algorithm2e}
\newcommand{\mycommfont}[1]{\small\ttfamily\textcolor{blue}{#1}}
\SetCommentSty{mycommfont}

%highlighting
\usepackage{mdframed}

\paperheight=11in
\topmargin=0in
\headheight=0in
\headsep=0in
\topskip=0in
\textheight=8.5in
\footskip=.5in

\paperwidth=8.5in
\oddsidemargin=.25in
\evensidemargin=.25in
\textwidth=6.0in
\parindent=.5in

\newtheorem{claim}{Claim}
\newcommand{\proof}[0]{{\bf proof:} }
\newcommand{\qed}[0]{\newline\noindent{\bf QED }}


\newcommand{\bra}[1]{\langle#1|}
\newcommand{\ket}[1]{|#1\rangle}
\newcommand{\av}[1]{\left\langle#1\right\rangle}
\newcommand{\pder}[2]{\frac{\partial#1}{\partial#2}}
\newcommand{\tr}[0]{{\rm tr }}
\newcommand{\beq}{\begin{equation}}
\newcommand{\eeq}{\end{equation}}
\newcommand{\bsub}{\begin{subequations}}
	\newcommand{\esub}{\end{subequations}}
\newcommand{\beqa}{\begin{eqnarray}}
\newcommand{\eeqa}{\end{eqnarray}}
\newcommand{\rarrow}[0]{\rightarrow}
\newcommand{\larrow}[0]{\leftarrow}
\newcommand{\uarrow}[0]{\uparrow}
\newcommand{\darrow}[0]{\downarrow}
\newcommand{\Rarrow}[0]{\Rightarrow}
\newcommand{\nRarrow}[0]{\nRightarrow}
\newcommand{\Larrow}[0]{\Leftarrow}
\newcommand{\nLarrow}[0]{\nLeftarrow}
\newcommand{\ul}[1]{\underline{#1}}
\newcommand{\ol}[1]{\overline{#1}}
\newcommand{\ZZ}[0]{ {\mathbb{Z}}}
\newcommand{\RR}[0]{{ \mathbb{R}} }
\newcommand{\CC}[0]{{ \mathbb{C}} }
\newcommand{\ground}{{}_{\stackrel{\stackrel{\displaystyle{\bot}}{-}}{.}}}
\newcommand{\norm}[1]{\parallel#1\parallel}
\newcommand{\eqdef}[0]{\;\;\stackrel{\text{def}}{=}\;\;} 
\newcommand{\sign}[0]{{\rm sign}}

\newcommand{\rva}[0]{{\ul{a}}}
\newcommand{\rvb}[0]{{\ul{b}}}
\newcommand{\rvc}[0]{{\ul{c}}}
\newcommand{\rvd}[0]{{\ul{d}}}
\newcommand{\rve}[0]{{\ul{e}}}
\newcommand{\rvf}[0]{{\ul{f}}}
\newcommand{\rvg}[0]{{\ul{g}}}
\newcommand{\rvh}[0]{{\ul{h}}}
\newcommand{\rvi}[0]{{\ul{i}}}
\newcommand{\rvj}[0]{{\ul{j}}}
\newcommand{\rvk}[0]{{\ul{k}}}
\newcommand{\rvl}[0]{{\ul{l}}}
\newcommand{\rvm}[0]{{\ul{m}}}
\newcommand{\rvn}[0]{{\ul{n}}}
\newcommand{\rvo}[0]{{\ul{o}}}
\newcommand{\rvp}[0]{{\ul{p}}}
\newcommand{\rvq}[0]{{\ul{q}}}
\newcommand{\rvr}[0]{{\ul{r}}}
\newcommand{\rvs}[0]{{\ul{s}}}
\newcommand{\rvt}[0]{{\ul{t}}}
\newcommand{\rvu}[0]{{\ul{u}}}
\newcommand{\rvv}[0]{{\ul{v}}}
\newcommand{\rvw}[0]{{\ul{w}}}
\newcommand{\rvx}[0]{{\ul{x}}}
\newcommand{\rvy}[0]{{\ul{y}}}
\newcommand{\rvz}[0]{{\ul{z}}}


\newcommand{\rvA}[0]{{\ul{A}}}
\newcommand{\rvB}[0]{{\ul{B}}}
\newcommand{\rvC}[0]{{\ul{C}}}
\newcommand{\rvD}[0]{{\ul{D}}}
\newcommand{\rvE}[0]{{\ul{E}}}
\newcommand{\rvF}[0]{{\ul{F}}}
\newcommand{\rvH}[0]{{\ul{H}}}
\newcommand{\rvK}[0]{{\ul{K}}}
\newcommand{\rvL}[0]{{\ul{L}}}
\newcommand{\rvN}[0]{{\ul{N}}}
\newcommand{\rvQ}[0]{{\ul{Q}}}
\newcommand{\rvR}[0]{{\ul{R}}}
\newcommand{\rvT}[0]{{\ul{T}}}
\newcommand{\rvU}[0]{{\ul{U}}}
\newcommand{\rvV}[0]{{\ul{V}}}
\newcommand{\rvW}[0]{{\ul{W}}}
\newcommand{\rvX}[0]{{\ul{X}}}
\newcommand{\rvY}[0]{{\ul{Y}}}
\newcommand{\rvZ}[0]{{\ul{Z}}}


\newcommand{\rvtheta}[0]{{\ul{\theta}}}


\newcommand{\cala}[0]{{\cal A}}
\newcommand{\calb}[0]{{\cal B}}
\newcommand{\calc}[0]{{\cal C}}
\newcommand{\cald}[0]{{\cal D}}
\newcommand{\cale}[0]{{\cal E}}
\newcommand{\calf}[0]{{\cal F}}
\newcommand{\calg}[0]{{\cal G}}
\newcommand{\calh}[0]{{\cal H}}
\newcommand{\cali}[0]{{\cal I}}
\newcommand{\calk}[0]{{\cal K}}
\newcommand{\call}[0]{{\cal L}}
\newcommand{\calm}[0]{{\cal M}}
\newcommand{\caln}[0]{{\cal N}}
\newcommand{\calo}[0]{{\cal O}}
\newcommand{\calp}[0]{{\cal P}}
\newcommand{\calr}[0]{{\cal R}}
\newcommand{\cals}[0]{{\cal S}}
\newcommand{\calt}[0]{{\cal T}}
\newcommand{\calu}[0]{{\cal U}}
\newcommand{\calv}[0]{{\cal V}}
\newcommand{\calw}[0]{{\cal W}}
\newcommand{\calx}[0]{{\cal X}}
\newcommand{\caly}[0]{{\cal Y}}
\newcommand{\calz}[0]{{\cal Z}}

\newcommand{\lam}[0]{\lambda}
\newcommand{\Lam}[0]{\Lambda}
\newcommand{\alp}[0]{\alpha}
\newcommand{\eps}[0]{\epsilon}
\newcommand{\s}[0]{\sigma}
\newcommand{\su}[0]{{\Sigma}}

\newcommand{\pp}[0]{\mathbb{P}}
\newcommand{\dbm}[0]{{
		[1,\partial_{b},\partial_{m}]
}}

\newcommand{\dg}[0]{{[1, \partial_{\theta_G}]}}
\newcommand{\dd}[0]{{[1, \partial_{\theta_D}]}}
\newcommand{\dgd}[0]{{[1, \partial_{\theta_G}, \partial_{\theta_D}]}}

\newcommand{\veca}[0]{{\vec{a}}}
\newcommand{\vecb}[0]{{\vec{b}}}
\newcommand{\vecc}[0]{{\vec{c}}}
\newcommand{\vecd}[0]{{\vec{d}}}
\newcommand{\vecf}[0]{{\vec{f}}}
\newcommand{\vech}[0]{{\vec{h}}}
\newcommand{\vecr}[0]{{\vec{r}}}
\newcommand{\vecs}[0]{{\vec{s}}}
\newcommand{\vecu}[0]{{\vec{u}}}
\newcommand{\vecx}[0]{{\vec{x}}}
\newcommand{\vecy}[0]{{\vec{y}}}
\newcommand{\vechy}[0]{{\vec{\hat{y}}}}
\newcommand{\vtheta}[0]{{\vec{\theta}}}



\newcommand{\cond}[0]{{\:\mathbf{|}\:}}
\newcommand{\mymathbf}[1]{#1}

\newcommand{\ranvec}[1]{\ul{\vec{#1}}}
\newcommand{\indi}[0]{\mathbbm{1}}
\newcommand{\sig}[0]{{\rm sig}}

\renewcommand{\labelitemii}{$\bullet$}

\newcommand{\tilx}[0]{{\tilde{x}}}
\newcommand{\tile}[0]{{\tilde{e}}}
\newcommand{\tilu}[0]{{\tilde{u}}}

\newcommand{\ucalm}[0]{\ul{\calm}}

\newcommand{\bool}[0]{\{0,1\}}

\newcommand{\argmin}{\mathop{\mathrm{argmin}}\limits}
\newcommand{\argmax}{\mathop{\mathrm{argmax}}\limits}  

\newcommand{\A}[0]{\wedge}
\newcommand{\V}[0]{\vee}

\newcommand{\rdart}[0]{\Rightarrow}
\newcommand{\ldart}[0]{\Leftarrow}
\newcommand{\rveps}[0]{\ul{\eps}}



%arguments phi1, phi2, phi3, e
\newcommand{\rbd}[4]{
\xymatrix@-1.3pc{
&#1\ar[d]&&#2\ar[d]
\\
&\rvx_1\ar[rr]
&&
\rvx_2
\ar `r[rd][rd]
\\
0\ar`u[u][ru]
\ar`d[dr][rrd]
&&&&\rvA\ar[r]&#4
\\
&&
\rvx_3
\ar `r[rru][rru]
\\
&&#3\ar[u]
}
}


\newcommand{\ruleone}[0]{
If $(\rvb. \perp \rva. 
|\rvr., \rvs.)$
in $\rho_{\rvr.} G$, then
$
P(b.|a., \rho\rvr.=r.,s.)=
P(b.|\rho\rvr.=r., s.)
$.
}

\newcommand{\ruletwo}[0]{
If $(\rvb.\perp \rva. |
 \rvr., \rvs.)$
in $\lam_{\rva.}\rho_{\rvr.} G$, 
then $
P(b.|\rho\rva.=a., \rho\rvr.=r., s.)=
P(b.|a., \rho\rvr.=r.,  s.)
$.
}

\newcommand{\rulethree}[0]{
If $
(\rvb. \perp
\rva.
| \rvr., \rvs.)$
in $
\rho_{\rva.-an(\rvs.)}
\rho_{\rvr.}G$,
 then
$
P(b.|\rho\rva.=a.,\rho\rvr.=r.,  s.)=
P(b.|\rho\rvr.=r., s.)$.
}

\newcommand{\bdoordef}[0]{
Suppose that we have access to data
that allows us to 
estimate a probability
distribution 
 $P(x., y., z.)$.
Hence, the variables
$\rvx., \rvy., \rvz.$ are 
ALL the observed (i.e, not hidden).
Then we say that the backdoor
$\rvz.$ satisfies the 
{\bf backdoor adjustment criterion}
relative to $(\rvx., \rvy.)$
if
\begin{enumerate}
\item 
All backdoor paths from $\rvx.$ 
to $\rvy.$ 
 are blocked by  
 $\rvz.$.
\item 
$\rvz. \cap de(\rvx.)=\emptyset$.
\end{enumerate}
}

\newcommand{\bdoorclaim}[0]{
If $\rvz.$ satisfies the 
backdoor criterion relative to
 $(\rvx., \rvy.)$, then

\beqa
P(y. | \rho \rvx.=x.)&=&
\sum_{z.} P(y.|x., z.)P(z.)
\\
&=&
\begin{array}{l}
\\
\\
\sum_{z.}
\end{array}
\xymatrix{
\rvz.=z.\ar[dr]
\\
\rvx.=x.\ar[r]&\rvy.
}
\eeqa
}

\newcommand{\fdoordef}[0]{
Suppose that we have access to data
that allows us to 
estimate a probability
distribution 
 $P(x., m., y.)$.
Hence, the variables
$\rvx., \rvm., \rvy.$ are 
ALL the observed (i.e, not hidden).
Then we say that the front-door
$\rvm.$ satisfies the 
{\bf front-door adjustment criterion} relative
 to $(\rvx., \rvy.)$
if
\begin{enumerate}
\item 
All directed paths from
$\rvx.$ to $\rvy.$ are intercepted by
(i.e., have a node in) $\rvm.$.
\item 
All backdoor paths from $\rvx.$ to
$\rvm.$ are blocked (by $\emptyset$).
\item 
All backdoor paths from
on $\rvm.$ to $\rvy.$
are blocked by $\rvx.$.
\end{enumerate}
}

\newcommand{\fdoorclaim}[0]{
If $\rvm.$ satisfies the 
front-door criterion 
relative to $(\rvx., \rvy.)$,
and $P(x.,m.)>0$, then

\beqa
P(y. | \rho \rvx.=x.)&=&\sum_{m.}
\underbrace{\left[\sum_{x.'}
P(y.|x'., m.)P(x'.)\right]}_
{P(y.|\rho \rvm.=m.)} 
\underbrace{P(m.|x.)}_
{P(m.|\rho \rvx.=x.)}
\\
&=&
\begin{array}{l}
\\
\\
\sum_{m., x.'}
\end{array}
\xymatrix{
&\rvx.=x.'\ar[dr]
\\
\rvx.=x.´\ar[r]
&\rvm.=m.\ar[r]&\rvy.
}
\eeqa
}


%Symmetry
\newcommand{\symrule}[0]{
$\rva\perp_P\rvb\implies \rvb\perp_P\rva$}

\newcommand{\symruleH}[0]{
$H(\rva:\rvb)=0\implies H(\rvb:\rva)=0$}

%Decomposition
\newcommand{\decrule}[0]{
$\rva\perp_P\rvb, \rvc\implies 
\rva\perp_P\rvb \text{ and } \rva\perp_P\rvc$}

\newcommand{\decruleH}[0]{
$H(\rva:\rvb, \rvc)=0\implies 
H(\rva:\rvb)=0 \text{ and } H(\rva:\rvc)=0$}

%Weak Union
\newcommand{\wearule}[0]{
$\rva\perp_P \rvb, \rvc \implies 
\rva\perp_P\rvb|\rvc\text{ and }\rva\perp_P\rvc|\rvb$}

\newcommand{\wearuleH}[0]{
$H(\rva:\rvb, \rvc)=0 \implies 
H(\rva:\rvb|\rvc)=0\text{ and }H(\rva:\rvc|\rvb)=0$}

%Contraction
\newcommand{\conrule}[0]{
$\rva\perp_P\rvb|\rvc\text{ and }\rva\perp_P \rvc
\implies \rva\perp_P \rvb, \rvc$}

\newcommand{\conruleH}[0]{
$H(\rva:\rvb|\rvc)=0\text{ and }H(\rva:\rvc)=0
\implies H(\rva:\rvb, \rvc)=0$}

%Intersection
\newcommand{\intrule}[0]{
$\rva\perp_P\rvb|\rvc, \rvd\text{ and }
\rva\perp_P \rvd|\rvc, \rvb\implies
\rva\perp_P \rvb,\rvd|\rvc$}

\newcommand{\intruleH}[0]{
$H(\rva:\rvb|\rvc, \rvd)=0\text{ and }
H(\rva:\rvd|\rvc, \rvb)=0\implies
H(\rva:\rvb,\rvd|\rvc)=0$}

\newcommand{\dotbarmu}[0]{{\cdot|\mu}}
\newcommand{\dotmu}[0]{{\cdot, \mu}}
\newcommand{\kbarmu}[0]{{k|\mu}}
\newcommand{\kmu}[0]{{k,\mu}}
\newcommand{\plusbarmu}[0]{{+|\mu}}
\newcommand{\plusmu}[0]{{+,\mu}}

\newcommand{\bnlearn}[0]{{\tt bnlearn\;}}

\newcommand{\sqsig}[0]{{[\sigma]}}

\newcommand{\misscellone}[0]{
\begin{array}{c}
\frac{1}{nsam}
P(x_0=0, x_2=0\cond x_1=1, \theta)
\\
\frac{1}{nsam}
P(x_0=0, x_2=1\cond x_1=1, \theta)
\\
\frac{1}{nsam}
P(x_0=1, x_2=0\cond x_1=1, \theta)
\\
\frac{1}{nsam}
P(x_0=1, x_2=1\cond x_1=1, \theta)
\end{array}
}

\newcommand{\misscelltwo}[0]{
\begin{array}{c}
\frac{1}{nsam}
P(x_1=0\cond x_0=0,x_2=1, \theta)
\\
\frac{1}{nsam}
P(x_1=1\cond x_0=0,x_2=1,  \theta)
\end{array}
}

\newcommand{\td}[0]{{\tilde{d}}}
\newcommand{\rvtd}[0]{{\ul{\tilde{d}}}}
\newcommand{\tx}[0]{{\tilde{x}}}
\newcommand{\rvtx}[0]{{\ul{\tilde{x}}}}

\begin{document}
\title{Goodness of Causal Fit}
\date{ \today}
\author{Robert R. Tucci\\
        tucci@ar-tiste.com}
\maketitle
\vskip2cm
\section*{Abstract}
In this paper, we propose a 
Goodness of Causal Fit (GCF) measure
which depends 
on Pearl ``do" interventions.
This is different
to a measure of Goodness of Fit (GF),
which does not use interventions.
Given a DAG set  $\calg$,
to find a good $G\in \calg$,
we propose plotting
$GCF(G)$ versus $GF(G)$
for all $G\in \calg$,
and finding a 
graph $G\in \calg$  with 
a large amount 
of both types of goodness.


\newpage
\section{Introduction}



Frequently,
when students
first encounter
Bayesian Networks (bnets)
and Causal Inference (CI)
(Refs.\cite{pearl-2013book},
\cite{bayesuvius}),
they experience serious doubts
about the usefulness of this
theory, because they believe
finding the underlying model (i.e., DAG)
for a physical situation is
too difficult or impossible.
I think
that part of the problem
is that these students
are assuming, perhaps
unconsciously,
that their
is a unique DAG
that fits Nature perfectly,
and a mind-boggling number
 of possibilities
to sift through to find that DAG.
Rather than looking
for a unique DAG,
I think a better strategy
is to write down
a set $\calg$ 
of likely DAGs,
and to calculate for 
each DAG in $\calg$,
 a measure called 
Goodness of Causal Fit (GCF).
Then use the DAGs with
the highest GCF scores.


The goal of this paper
is to propose a GCF measure.
Such a measure is of course
not unique,
and someone may propose a measure better
than ours in the future.

When inventing a GCF measure,
it is important to keep
in mind the First Dictum\footnote{
This is just my whimsical name for it.} of CI: The data is model-less.
 In the First Dictum,
when we say ``data", we are referring to what is commonly
 called a dataset. A dataset is a table of data, where all
 the entries of each column have the same units, and 
measure a single feature, and each row refers to one
 particular sample or individual. Datasets are particularly 
useful for estimating probability distributions and for 
training neural nets. In the First Dictum, when we say ``model", we are 
referring to a DAG (directed acyclic graph) or a bnet
 (Bayesian Network= DAG + probability table 
for each node of DAG).

You can try to derive a model from a dataset, 
but you'll soon find out that you can only go so far.
The process of finding a {\it partial} model from a dataset 
is called structure learning (SL).  SL can be done quite
 nicely with Marco Scutari's open source program 
{\tt bnlearn} (Ref\cite{bnlearn}).
The problem is that SL often cannot 
narrow down the model to a single one. It finds an undirected graph (UG), 
and it can determine the direction of some of the arrows in the UG, 
but it is often incapable, for well understood 
fundamental ---not just technical--- reasons,
 of finding the direction of ALL the arrows of the UG. 
So it often fails to fully specify a DAG model.

Let's call the ordered pair (dataset, model) a 
{\bf data SetMo}.
 Then what the First Dictum is saying is that a dataset 
is model-free or model-less (although sometimes one can 
find a partial model hidden in there). 
A dataset is not a data SetMo.

Graphs
which contain both directed 
and undirected edges
are called 
{\bf partially directed (PD) graphs}.
Let $\calg(G_{pd})$
be a DAG set $\calg$
which 
is generated by a PD graph $G_{pd}$
by giving directions to all 
undirected edges of $G_{pd}$
in all possible ways.
We will refer to 
 $\calg(G_{pd})$ as 
the {\bf DAG set generated by $G_{pd}$}.
and to any $\calg'\subset 
\calg(G_{pd})$ as a {\bf DAG set partially 
generated by $G_{pd}$.}

{\tt bnlearn} takes
a dataset as input
and returns a PD graph
$G_{pd}$.
Sets like $\calg(G_{pd})$
are nice candidates,
although not the only possible
candidates, for
a DAG set 
 $\calg$
to which one can apply 
our GCF measure.


It's clear that any measure
of GCF will have to
involve interventions
such as the ``do" intervention
invented by Pearl et al for CI.
Without interventions like ``do",
it is impossible
to distinguish causally the
DAGs in a set $\calg(G_{pd})$.

Henceforth, random variables
will be indicated by underlining.

\section{Goodness of Fit}
Before trying to
define a GCF measure,
it
is instructive to review 
the closely related, well established, measures
of Goodness of Fit (GF). 



Consider 
two
probability distributions
$PO(x)$ and $PE(x)$,
where $x\in S_\rvx$.
By a GF measure, we mean 
a measure of the 
difference between 
$PO$ and $PE$.
Usually $PO$ is the
observed probability distribution and 
$PE$ is the expected, theoretical one.


Three popular
measures of
the difference between $PO$ and $PE$
are:
\begin{enumerate}
\item
The
{\bf Kullback-Liebler divergence}:
\beq
D_{KL}(PO\parallel PE) =
\sum_{x\in S_\rvx}
PO(x)\ln \frac{PO(x)}{PE(x)}
\;.
\eeq
\item
The
{\bf Pearson divergence}
(aka {\bf Pearson Chi-squared test statistic}):
\beq
D_{\chi^2}(PO\parallel PE)=
\sum_{x\in S_\rvx}
\frac{[PO(x)-PE(x)]^2}{PE(x)}
=
\sum_{x\in S_\rvx}
\frac{PO^2(x)}{PE(x)}-1
\;.
\eeq

It's easy to show that
if $\left|\frac{PO(x)}{PE(x)}-1\right|<<1$
for all $x\in S_\rvx$, then

\beq
D_{KL}(PO\parallel PE)\approx 
D_{\chi^2}(PO\parallel PE)
\eeq

\item
The {\bf Euclidean distance squared}:

\beq
D_E(PO,PE)=
\sum_{x\in S_\rvx}
[PO(x)-PE(x)]^2
\eeq
\end{enumerate}
Note that of these 3 measures,
only $D_E(PO, PE)$ is symmetric 
in $PO$ and $PE$.


Given any bnet $G$
with full probability
distribution
\footnote{We define
$x.$
to be a vector
with components $x_i$}
  $P_G(x.)$
and a
probability distribution\footnote{
Empirical distributions will 
be denote by $P$ with a tilde over it.}
$\tilde{P}(x.)$
derived empirically from a dataset,
let

\beqa
D(G)
&=&
\sum_{x.}
\tilde{P}(x.)\ln 
\frac{\tilde{P}(x.)}
{P_{G}(x.)}
\\
&=&
D_{KL}(
\tilde{P}(\rvx.)\parallel P_{G}(\rvx.))
\eeqa
We define {\bf Goodness of Fit (GF)}
of the bnet $G$ by

\beq
GF(G)=\ln \frac{1}
{D(G)}
\eeq

\section{GCF example 1}

\begin{figure}[h!]
$$
\begin{array}{ccc}
\xymatrix{
&\rvb\ar@[red]@{-}[ld]\ar[dr]
\\
\rva\ar[rr]&&\rvz
}
&
\xymatrix{
&\rvb\ar[dl]\ar[dr]
\\
\rva\ar[rr]&&\rvz
}
&
\xymatrix{
&\rvb\ar[dr]
\\
\rva\ar[ur]\ar[rr]&&\rvz
}
\\
G_{pd}&G_1&G_2
\end{array}
$$
\caption{$\calg(G_{pd})=\{G_1, G_2\}$.
The partially directed graph $G_{pd}$
generates the DAGs $G_1$ and $G_2$
by giving directions to
all undirected edges of $G_{pd}$
in
all possible ways.
(In this case, there is only one
undirected edge in $G_{pd}$.) }
\label{fig-ob-eq-1}
\end{figure}

For the first example of
our measure of GCF,
we consider 
$\calg(G_{pd})=\{G_1, G_2\}$
given by Fig.\ref{fig-ob-eq-1}.
We will assume as follows:

\begin{itemize}
\item
First, we assume that we have collected
a dataset from which wee have
extracted a full empirical
distribution
$\tilde{P}(z, a,b)$.
From $\tilde{P}(z, a,b)$,
we assume that the following
have been calculated.
$\tilde{P}(z,b|a)$
$\tilde{P}(z, a|b)$
$\tilde{P}(a)$, $\tilde{P}(b)$.
\item
Second, we assume that a
dataset has been collected
 for which $\rva$ was held
fixed to each of
the possible values
$a\in S_\rva$ of $\rva$.
Furthemore, we assume
that the distribution
$\tilde{P}(z, b|do(\rva)=a)$
has been extracted from that dataset.
\item
Third, we assume that a
dataset has been collected
 for which $\rvb$ was held
fixed to each of
the possible values
$b\in S_\rvb$ of $\rvb$.
Furthemore, we assume that
the distribution
$\tilde{P}(z, a|do(\rvb)=b)$
has been extracted
from that dataset.
\end{itemize}
We will refer to
$\tilde{P}(z, b|do(\rva)=a)$
and 
$\tilde{P}(z, a|do(\rvb)=b)$
as {\bf empirical do-probabiity distributions}.


Now define
\beqa
D_a&=&
\sum_{z,b}\tilde{P}(z,b|a)
\ln
\frac{\tilde{P}(z,b|a)}
{\tilde{P}(z,b|do(\rva)=a)}
\\
&=&D_{KL}(\tilde{P}(\rvz, \rvb|a)
\parallel \tilde{P}(\rvz, \rvb|do(\rva)=a))
\\
D_\rva &=& \sum_a \tilde{P}(a) D_a = E_a[D_a]
\eeqa
and

\beqa
D_b
&=&
D_{KL}(\tilde{P}(\rvz, \rva|b)
\parallel \tilde{P}(\rvz, \rva|do(\rvb)=b))
\\
D_\rvb 
&=&
\sum_a \tilde{P}(b) D_b = E_b[D_b]
\;.
\eeqa
We will
refer to $D_\rva$ and $D_\rvb$
as {\bf do-divergences}.

Note that
\beq
D_a(G_2)=0
\text{ for all $a$ so }
\underbrace{D_\rva(G_2)}_0
\leq D_\rvb(G_2)
\eeq
and

\beq
D_b(G_1)=0
\text{ for all $b$ so }
D_\rva(G_1)\geq 
\underbrace{D_\rvb(G_1)}_0
\;.
\eeq

If $D_\rva\geq D_\rvb$ then
$\rva\larrow \rvb$,
and if $D_\rva\leq D_\rvb$,
then $\rva\rarrow \rvb$.
Thus, the arrow and the 
inequality sign point
 in opposite directions.
(Alternatively, just remember that
the arrow points to the larger of 
the two $D$'s).

If $D_\rva\leq  D_\rvb$, then define
$GCF(G_1)=-1$ and $GCF(G_2)=1$.

If $D_\rvb\leq  D_\rva$, then define
$GCF(G_1)=1$, $GCF(G_2)=-1$.



\section{GCF example 2}

\begin{figure}[h!]
$$
\xymatrix{
&\rvx_1
\ar@[red]@{-}[dl]\ar@[red]@{-}[dr]
\\
\rvx_2\ar[dr]
&&\rvx_3\ar[dl]
\\
&\rvx_4\ar[d]
\\
&\rvx_5
\\
&G_{pd}
}
\;\;\;\;\;
\xymatrix{
&\rvx_1
\ar[dl]
\ar[dr]
\\
\rvx_2\ar[dr]
&&\rvx_3\ar[dl]
\\
&\rvx_4\ar[d]
\\
&\rvx_5
\\
&G_1
}
\;\;\;\;\;
\xymatrix{
&\rvx_1
\ar[dr]
\\
\rvx_2\ar[dr]\ar[ur]
&&\rvx_3\ar[dl]
\\
&\rvx_4\ar[d]
\\
&\rvx_5
\\
&G_2
}
\;\;\;\;\;
\xymatrix{
&\rvx_1
\ar[dl]
\\
\rvx_2\ar[dr]
&&\rvx_3\ar[dl]\ar[ul]
\\
&\rvx_4\ar[d]
\\
&\rvx_5
\\
&G_3
}
$$
\caption{$\calg=\{G_1, G_2, G_3\}$.
$\calg$ is a set of observationally
equivalent (OE) graphs. 
These are graphs that have the
same value for GF, and
are therefore indistinguishable
from GF alone. For more info about 
OE graphs, see Chapter
entitled ``Observationally Equivalent DAGs"
in Ref.\cite{bayesuvius}.
Note that $\calg(G_{pd})$
includes one more DAG,
the one in which node $\rvx_1$
is a collider.}
\label{fig-ob-eq-2}
\end{figure}

For the second example of
our measure of GCF,
consider 
$\calg=\{G_1, G_2, G_3\}$
given by Fig.\ref{fig-ob-eq-2}.

If we evaluate
the do-divergences
$D_{\rvx_2}$, $ D_{\rvx_1}$ and $D_{\rvx_3}$,
their relative
sizes will be determined
from the empirical
do-probability distributions.
For instance,
suppose their
sizes are ordered as follows:



\beq
D_{\rvx_2}\leq  D_{\rvx_1} \leq  D_{\rvx_3}
\;.
\eeq

Define the distance
\beq
d_{\rvb,\rva}=
|D_\rvb-D_\rva|
\eeq
for any two do-divergences
$D_\rva$ and $D_\rvb$.

If we abbreviate $D_{\rvx_j}$ by $D_j$, 
we can define the GCF for each of
the graphs in $\calg$ by:

\begin{subequations}
\label{eq-gcf-example2}
\beq
GCF(G_1)=
\frac{
-d_{2,1} + d_{1,3}
}
{d_{2,1} + d_{1,3}}
\eeq

\beq
GCF(G_2)=
\frac{
d_{2,1} + d_{1,3}
}
{d_{2,1} + d_{1,3}}=1
\eeq

\beq
GCF(G_3)=
\frac{
-d_{2,1} - d_{1,3}
}
{d_{2,1} + d_{1,3}}=-1
\eeq
\end{subequations}

\section{GCF in general}
Eqs.(\ref{eq-gcf-example2})
are a special case of
the following formulas.

For any two do-divergences
$D_\rva$ and $D_\rvb$,
define the distance:
\beq
d_{\rvb,\rva}=
|D_\rvb-D_\rva|
\;.
\eeq
Define 
the {\bf edge sign function} 
for any $G_i\in\calg$ by
\beq
\s_{\rva\rarrow \rvb}(G_i)=
\left\{
\begin{array}{ll}
+1&\text{ if arrow connecting
$\rva$ and $\rvb$ in $G_i$
points
from $\rva$ to $\rvb$.}
\\
-1&\text{ otherwise}
\end{array}
\right.
\eeq
Finally, suppose that
$\calg$
is either partially
or fully
generated by
a PD graph $G_{pd}$
with undirected edges
$\{\rva_k\text{---}
\rvb_k\}_{k=0,1, \ldots, nk-1}$.
Then 
define the GCF of 
graph
$G_i\in \calg$ by

\beq
GCF(G_i)= 
\frac{
\sum_{k=0}^{nk-1}
\s_{\rva_k\rarrow \rvb_k}(G_i)
d_{\rva_k, \rvb_k}
}
{
\sum_{k=0}^{nk-1}
d_{\rva_k, \rvb_k}
}
\;.
\eeq
Note that
$-1\leq GCF(G_i)  \leq 1$.



So far, we 
have applied our measure of
GCF to DAG sets $\calg$
which are 
either fully or
partially generated
by a PD graph $G_{pd}$.
If the DAG set $\calg$ 
contains only one DAG,
define the GCF of that DAG
to be 1, because all
arrows are in the right
direction as far as we know.
We can also
apply  our GCF measure
to {\bf composite DAG sets} $\calg$ which equal the
union of either
fully or
partially 
generated DAG sets,
and singleton sets 
which contain
only one DAG.


So let $\calg$
be a composite DAG set.
Our GCF measure
is not enough to
decide the best 
possible $G$ in $\calg$,
because there might 
be several graphs with $GCF=1$.
For this reason,
we recommend
plotting $GCF(G)$
versus $GF(G)$
for all $G\in\calg$.
Then  choose a $G$ with a
large amount
of both types of goodness.
It might even be
advantageous to average over
a small subset of DAGs in  $\calg$
that have large amounts of both
types of goodness.
This would be akin to  averaging
over decision trees to get 
a random forest.

A plot of 
$GCF(G)$
versus $GF(G)$
agrees with the spirit of
the First 
Dictum and data setmos,
because in those too,
we acknowledge a separation between the
dataset (GF)  and model (GCF) degrees of freedom.

\bibliographystyle{plain}
\bibliography{references}
\end{document}
